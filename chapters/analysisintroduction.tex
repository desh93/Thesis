\chapter{Analysis overview}\label{chap:analyintro}

% \section{c}

Historically signatures with dilepton final states have been central in shaping the SM, from discoveries of new particles~\cite{PhysRevLett.33.1404, PhysRevLett.33.1406,1977PhRvL..39..252H,1983398,BAGNAIA1983130}, through many precision measurements~\cite{ALEPH:2005ab,Aad:2016zzw,Aad:2016izn,Sirunyan:2018swq}, and in searches for new BSM physics~\cite{Aad:2019fac,Sirunyan:2018exx,Sirunyan:2018ipj,EXOT-2016-05}. Traditionally, the dilepton channel has been used due to it's clean and fully reconstructable experimental signature with excellent detector efficiency. The sections outlines a novel search for new phenomena in final states with two electrons (\ee) or two muons (\mumu) at \SI{139}{\femto\barn^{-1}} of data collected in \protonproton collisions at the LHC at a $\sqrt{s}=\SI{13}{\tera\electronvolt}$. This analysis compliments the search for heavy resonances~\cite{Aad:2019fac} by using the same dataset and selection criteria. These non-resonant signatures have a broad deviation in the tails of the smoothly falling invariant mass spectrum, where the main background is the irreducible Z/$\gamma^*$ (Drell-Yan, DY) process. The results are provided in a model-independent format, which are then interpreted in the context of the frequently tested benchmark CI models. A detailed overview for the theoretical motivation for contact interaction searches is given is~\cref{chap:SM}. The \ee and \mumu invariant mass spectrum is chosen as the main observable in the analysis as it offers the best discriminating power against signal and background for non-resonant signals like CIs. 

A number of changes has been introduced in this analysis with respect to previous ATLAS results~\cite{EXOT-2016-05}. This is the first non-resonant dilepton search at the LHC to use a background estimate from data-driven fit method, where the background at high invariant mass is estimated from a low-mass control region (CR), utilising an extrapolation procedure. Estimation of the background using from the data considerably reduces the effect of the impact from the size of the MC samples or potential MC mismodelling. Additionally, the search is performed in a high mass single-bin signal region (SR), where both the regions and the function choice is optimised maximise sensitivity of observing a CI process. 

This chapter describes the analysis strategy and results from the Run-2 ATLAS search for non-resonant phenomena using \SI{139}{\femto\barn^{-1}} of data at a $\sqrt{s}=\SI{13}{\tera\electronvolt}$. \cref{fig:nonres:intro:analysismodel} outlines the analysis model of the search. The object and event selection is described in ~\cref{chap:eventsel}. The selection is validated in terms of data/MC comparisons, where the MC sample used are outlines in ~\cref{chap:datamc}. The "transfer function" described in ~\cref{sec:datamc:transfer} are a parametrisation of the detector mass resolution introduced in the analysis to overcome the limitations of low statistics in MC, which are used to validate the functional fit. The background estimation procedure is described in ~\cref{chap:bkgmodel} and the final results are presented in ~\cref{chap:results}.

\begin{figure}[h]
    \centering
    \includegraphics[width=\mediumfigwidth]{figures/analysis/introduction/AnalysisOverview.pdf}
    \caption[Analysis model of the full Run-2 dilepton non-resonant search]{Analysis model of the full Run-2 dilepton non-resonant search.}
    \label{fig:nonres:intro:analysismodel}
\end{figure}
\clearpage