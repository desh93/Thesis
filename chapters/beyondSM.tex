\chapter{Beyond the Standard Model}\label{chap:bsm}

The Standard Model of particle physics has had tremendous success in describing the majority of the observed phenomena so far. However, there still remain several important unanswered questions from experimental signatures, that do not fit within the framework of the SM. These signatures hint towards physics beyond the Standard Model (BSM). This thesis focuses on a search for contact interaction (CI) models in dilepton final states. 

This chapter will briefly outline some of the outstanding questions in the SM that motivates BSM searches. Additionally, the four-fermion contact interaction will be summarised. 

\section{Motivation}
The SM mechanism to generate neutrino masses relies on unobserved right-handed neutrinos, and require large parameter turning to describe the size of the observed neutrinos masses. The search for answers to these questions is one such motivation for BSM theories. Many BSM theories attempt to explain these two phenomena. 

Another such phenomena results from global fits to the cosmic microwave background (CMB) radiation has indicated that the baryonic matter described by the SM only makes up 5\% of the total mass of the universe~\cite{Bennett_2013,2014Plank}. Measurements of rotation curves of galaxies and gravitational lensing~\cite{roos2010dark} indicate existence of dark matter, which constitutes 27\% of the universe. In many theories, dark matter does not interact via the strong interaction, is not electrically charged, or interacts with electromagnetic radiation. Therefore, the only possible candidate in the SM is the neutrino. The remainder of the universe is expected to be consist of dark energy, inferred through observations of the expansion of the universe~\cite{Peebles_2003}. 

An additional observed shortcoming of the SM, and one of the strongest motivations for BSM models, is a result of an internal inconsistency of the SM itself. The observed Higgs mass is $\sim \SI{100}{\giga\electronvolt}$, and is a result of the electroweak symmetry breaking scale. There is a large difference between this and the Plank scale ($\sim 10^19~\SI{}{\giga\electronvolt}$), with no other scale present until then. There Plank scale is where quantum gravitational effects become dominant. Therefore, the Higgs mass is unprotected from large vacuum fluctuations and can result in corrections proportional to corrections from one-loop diagrams of virtual particles, where an infinite number of corrections could exist. In order to counter these corrections, the measured Higgs mass needs to be finely tuned~\cite{Giudice_2008}. This fine tuning motivates the existence of TeV scale physics, which would cancel some of the divergent corrections. \emph{supersymmetry} (SUSY) is one possible theory that attempts to explain the observed hierarchy problem~\cite{MARTIN_1998}. SUSY introduces \emph{superpartners} for each fermion and an extra spin one boson in the SM that have opposite spin to cancel the loop corrections on the Higgs mass. However, SUSY has not yet been observed, thus introducing a new SUSY breaking scale, which results in some fine tuning. An alternative approach to mitigate the hierarchy problem is by the introduction of some large extra dimensions, where gravity is allowed to propagate in new dimensions~\cite{Arkani_Hamed_1998}. 

\section{Brief overview of effective field theory}

