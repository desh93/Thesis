\chapter{Beyond the Standard Model}\label{chap:bsm}

The Standard Model of particle physics has had tremendous success in describing the majority of the observed phenomena so far. However, there still remain several important unanswered questions from experimental signatures, that do not fit within the framework of the SM. These signatures hint towards physics beyond the Standard Model (BSM). This thesis focuses on a search for contact interaction (CI) models in dilepton final states. 

This chapter will briefly outline some of the outstanding questions in the SM that motivates BSM searches. Additionally, the four-fermion contact interaction will be summarised. 

\section{Motivation}
The SM mechanism to generate neutrino masses relies on unobserved right-handed neutrinos, and require large parameter turning to describe the size of the observed neutrinos masses. The search for answers to these questions is one such motivation for BSM theories. Many BSM theories attempt to explain these two phenomena. 

Another such phenomena results from global fits to the cosmic microwave background (CMB) radiation has indicated that the baryonic matter described by the SM only makes up 5\% of the total mass of the universe~\cite{Bennett_2013,2014Plank}. Measurements of rotation curves of galaxies and gravitational lensing~\cite{roos2010dark} indicate existence of dark matter, which constitutes 27\% of the universe. In many theories, dark matter does not interact via the strong interaction, is not electrically charged, or interacts with electromagnetic radiation. Therefore, the only possible candidate in the SM is the neutrino. The remainder of the universe is expected to be consist of dark energy, inferred through observations of the expansion of the universe~\cite{Peebles_2003}. 

An additional observed shortcoming of the SM, and one of the strongest motivations for BSM models, is a result of an internal inconsistency of the SM itself. The observed Higgs mass is $\sim \SI{100}{\giga\electronvolt}$, and is a result of the electroweak symmetry breaking scale. There is a large difference between this and the Plank scale ($\sim 10^19~\SI{}{\giga\electronvolt}$), with no other scale present until then. There Plank scale is where quantum gravitational effects become dominant. Therefore, the Higgs mass is unprotected from large vacuum fluctuations and can result in corrections proportional to corrections from one-loop diagrams of virtual particles, where an infinite number of corrections could exist. In order to counter these corrections, the measured Higgs mass needs to be finely tuned~\cite{Giudice_2008}. This fine tuning motivates the existence of TeV scale physics, which would cancel some of the divergent corrections. \emph{supersymmetry} (SUSY) is one possible theory that attempts to explain the observed hierarchy problem~\cite{MARTIN_1998}. SUSY introduces \emph{superpartners} for each fermion and an extra spin one boson in the SM that have opposite spin to cancel the loop corrections on the Higgs mass. However, SUSY has not yet been observed, thus introducing a new SUSY breaking scale, which results in some fine tuning. An alternative approach to mitigate the hierarchy problem is by the introduction of some large extra dimensions, where gravity is allowed to propagate in new dimensions~\cite{Arkani_Hamed_1998}. 

\section{Brief overview of effective field theory}
An effective field theory (EFT) is a low-energy approximation of a higher-energy QFT. With the use of an EFT indirect effects of particles with masses well beyond what is currently able to be experimentally probed can be quantified. They are based on the property of decoupling~\cite{manohar2018introduction}, where it refers to the screening of high-energy phenomena for interactions at lower energy scales. Using the expansion of the propagator of a massive particle with mass m, 
\begin{equation}
    \label{eq:propogator}
    \begin{aligned}
        \frac{1}{q^2 - m^2} = \frac{1}{m^2}(1 + \frac{q^2}{m^2} + ..),
     \end{aligned}
\end{equation}
it is clear that when $m^2 \gg q^2$, the propagation of the particle is suppressed. This is also apparent from the uncertainty principle, where the range of the force reduces as the mass of the particle grows large. Therefore, a mass of a mediator can be taken large enough that the propagator connecting interaction vertices is reduced to a point, resulting in a contact interaction. This results in the reduction of many theories into smaller sets of EFT operators. This allows the construction of an effective Lagrangian describing the interactions at lower energies only in terms of SM fields. 

Historically the \emph{top-down} approach was prevalent, where a consistent theory valid at very high scales is defined, and observations are constructed within that framework. However, due to lack of observations of new BSM heavy states at the LHC validating such models, the focus has shifted to a \emph{bottom-up} model building. The bottom-up approach considers the SM as the low-energy remobilisable gauge theory within in a theory at higher energy scales $\Lambda_{\mathrm{BSM}}$. Using the bottom-up approach, higher order operators can be added to the SM and a general effective Lagrangian can be written as: 
\begin{equation}
    \label{eq:eftLagrangian}
    \begin{aligned}
        \mathcal{L}_\mathrm{EFT} &=  \mathcal{L}_\mathrm{SM} +  \mathcal{L}_\mathrm{D},\\
        \mathcal{L}_\mathrm{D}   &= \sum_i \frac{c_i}{\Lambda_{\mathrm{BSM}}^{D-4}}\mathcal{O}^{(D)}_i.
     \end{aligned}
\end{equation}
$\mathcal{L}_\mathrm{SM}$ is the dimension four SM Lagrangian, $\Lambda_{\mathrm{BSM}}$ is the mass scale for BSM, and $\mathcal{O}_i$ are the dimension d effective operators. The $c_i$ are dimensionless couplings that specify the strength of the BSM interactions, known as \emph{Wilson coefficients}. There is only one dimension five operator that is gauge invariant, but is irrelevant for the work presented in this thesis. The work presented in this thesis is restricted to dimension six. 
\subsubsection{Fermi theory}
The classic example of an EFT approach is the Fermi theory of low-energy weak interactions. The full UV theory is the SM, and it is matched to the EFT by considering a theory valid at small momenta compared to $m_{Z,W}$~\cite{manohar2018introduction}. Muon decay, $\mu \rightarrow e\nu_\mu\bar{\nu}_e$, is considered as an example, where the SM energy scale ($\Lambda_{\mathrm{BSM}}\sim\SI{80.8}{\giga\electronvolt}$) is much higher than that of the interactions ($m_\mu \sim \SI{100}{\giga\electronvolt}$). 

In the SM, the decay amplitude is given by
\begin{equation}
    \label{eq:smmuon}
    \begin{aligned}
        \mathcal{M} =  - \left( \frac{g}{\sqrt{2}} \right)^2
        [\bar{\nu}_\mu\frac{1}{2}\gamma^\mu(1-\gamma^5)\mu]
        \left[ \frac{g_{\mu\nu} - q_\mu q_\nu/ m_W^2}{q^2 - m_W^2}  \right]
        [\bar{e}\frac{1}{2}\gamma^\mu(1-\gamma^5)\nu_e].
     \end{aligned}
\end{equation}
Considering the low-energy limit where $m_W^2 \gg q^2$, the propagator can be expanded as shown in \cref{eq:propogator}. Retaining only the first term of te expansion, the amplitude can now be written as
\begin{equation}
    \label{eq:smmuon1}
    \begin{aligned}
        \mathcal{M} =  \frac{g^2}{8 m_w^2}g_{\nu\mu}
        [\bar{\nu}_\mu\gamma^\mu(1-\gamma^5)\mu]
        [\bar{e}\gamma^\mu(1-\gamma^5)\nu_e].
     \end{aligned}
\end{equation}

In the Fermi effective theory, weak interactions are described by four-fermion contact interaction, with coupling $c/\Lambda^2$, where c is the wilson coefficient and $\Lambda$ is the scale of the EFT. This results in the amplitude
\begin{equation}
    \label{eq:eftmuon}
    \begin{aligned}
        \mathcal{M} =  \frac{c}{4\Lambda^2}g_{\nu\mu}
        [\bar{\nu}_\mu\gamma^\mu(1-\gamma^5)\mu]
        [\bar{e}\gamma^\mu(1-\gamma^5)\nu_e].
     \end{aligned}
\end{equation}
Hence, by comparing \cref{eq:smmuon1,eq:eftmuon}, the amplitude from the SM and Fermi theory can be matched by setting $\Lambda = m_W^2$ and $ c - g^2/2$. The Fermi constant, $G_F$, is related to EFT coupling by 
\begin{equation}
    \label{eq:fermiconstant}
    \begin{aligned}
        \frac{G_F}{\sqrt{2}} = \frac{c}{4\lambda^2} = \frac{g_W^2}{8m_W^2}. 
     \end{aligned}
\end{equation}

\section{Four-fermion contact interactions}
A possible solution to the hierarchy problem in the SM is the existence of constituents of quarks and leptons, known as \emph{preons}~\cite{Eichten:1984eu,Eichten:1983hw}. If quarks and leptons are composite, with at least one common constituent, the interaction of these constituents could manifest itself through an effective four-fermion CI at energies well below the compositeness scale.

The list of EFT operators was first classified in 1986~\cite{Buchmuller:163116}, and there existed 80 operators for each flavour. There were several ways of reducing the operators, as transformations between operators were possible. The \emph{Warsaw basisis}~\cite{Grzadkowski_2010} is the first complete and non-redundant set of dimension-6 operators that were proposed. This thesis focuses on a broad class of lepton-quark four-fermion CIs parameterised by the dimension six operator~\cite{de_Blas_2013}
\begin{equation}
    \label{eq:CIop}
    \begin{aligned}
        \mathcal{O}_{\ell q} = (\bar{\ell}\gamma^\mu \ell)(\bar{q}\gamma^\mu q),
     \end{aligned}
\end{equation}
where $\ell$ and \emph{q} are the SM lepton and quark doublets. Using this operator the Lagrangian is defined as 
\begin{equation}
\label{eq:CIlagrangian}
\begin{aligned}
\mathcal L = \frac{g^2}{2\Lambda^2} \{
    &\eta_{LL}\left(\bar{q}_L\gamma_{\mu}q_L\right)\left(\bar{\ell}_L\gamma^{\mu}\ell_L\right) + \eta_{RR}\left(\bar{q}_R\gamma_{\mu}q_R\right)\left(\bar{\ell}_R\gamma^{\mu}\ell_R\right) + \\
    &\eta_{LR}\left(\bar{q}_L\gamma_{\mu}q_L\right)\left(\bar{\ell}_R\gamma^{\mu}\ell_R\right) + \eta_{RL}\left(\bar{q}_R\gamma_{\mu}q_R\right)\left(\bar{\ell}_L\gamma^{\mu}\ell_L\right)\},
\end{aligned}
\end{equation}
where $g$ is a coupling constant chosen such that $g^2/4\pi = 1$, $\gamma$ are the Dirac matrices and the spinors $\mathcal{L}_{L,R}$ are the left-handed and right-handed fermion fields, respectively. The parameters $\eta_{ij}$, where $i$ and $j$ are $L$ or $R$, define the chiral structure (left or right) of the new interaction. Specific models are chosen by assigning the parameters to be $-1$, $0$ or $+1$. The sign of the $\eta_{ij}$ determines whether the interference with SM is constructive ($\eta_{ij} = -1$) or destructive constructive ($\eta_{ij} = +1$). The cross section for the process $q\bar{q} \rightarrow \ell\ell$ in the presence of such contact interactions is given by
\begin{eqnarray}
    \frac{d\sigma}{dm_{\ell\ell}} = \frac{d\sigma_\textrm{DY}}{dm_{\ell\ell}} - \eta_{ij}\frac{F_\textrm{I}}{\Lambda^2} + \frac{F_\textrm{C}}{\Lambda^4},
    \label{eq:cross_section_CI}
\end{eqnarray}
where the first term accounts for the $q\bar{q} \rightarrow Z/\gamma* \rightarrow \ell\ell$ Drell-Yan (DY) process. The second term corresponds to the interference between the DY and CI processes, and the third term give the pure CI process. $F_I$ and $F_C$ are functions of differential cross-section with repect to $m_{\ell\ell}$, and do not depend of $\Lambda$~\cite{Eichten:1984eu}. The experimental signature that is searched for in this thesis is a non-resonant enhancement of the expected dilepton $m_{\ell\ell}$ distribution at high mass. 

The results presented in this thesis focuses on the aforementioned CI model. The results are also provided in a model independent context~\cref{sec:results:reinterp}, which can be used to reinterpret into other models. These can be used to interpret the results in the context of any arbitrary quark-lepton CI model not included in the above model~\cite{de_Blas_2013}.

