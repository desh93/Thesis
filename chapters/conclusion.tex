\chapter{Conclusion and outlook}\label{chap:conclusion}

A search for non-resonant signals, with an interpretation of contact interactions, has been performed in dilepton final states using 139 fb$^{-1}$ of\protonproton collision data collected by the ATLAS experiment during Run~2 of the LHC at $\sqrt{s}=13$~TeV. The analysis uses novel approach to estimate the background contribution, where a data-driven extrapolation procedure is utilised. Two SRs each for the electron and muon channels are used to search for constructive and destructive interference models, resulting in four SRs in the analysis. The statistical uncertainty on the fit dominates the search in all of the SRs considered. Therefore, indicating that the search is statistics dominated. However, The unintuitive PDF uncertainties have a much smaller impact on the analysis compared to previous iterations. 

The background expectation is compared to the data, and possible deviations have been quantified in terms of significances using a profile-likelihood ratio test. An small excess is observed in the electron constructive SR corresponding to a significance of $1.28\sigma$. The other SRs observe a small deficit ranging between $-0.99\sigma$ and $ -0.19\sigma$. The significances show that the data are compatible with the background expectation, indicating that there is no new physics found. 

Since no significant excess or deficit is observed, upper limits on the number of signal events, as well as lower limits on the CI energy scale $\Lambda$ are set. Additionally, the acceptance times efficiency values for the CI signal models are also provided to aide in reinterpretation of the limits on the number of signal events. The electron and muon channel results are statistically combined to provide limits on the combined dilepton channel. The strongest limits are set on the combined-channel LL constructive model, where the observed and expected limits exclude this model for $\Lambda$ up to 35.8 and 37.6~\SI{}{\tera\electronvolt} at 95\% CL, respectively. 

The expected sensitivity of the analysis as increased by \SI{10}{\tera\electronvolt} compared to the previous iteration of the analysis performed with $\SI{36.1}{\femto\barn}^{-1}$ of data. Due to the excess observed in the electron constructive signal region, the resulting  observed limits show a smaller increase compared to the previous analysis. Additionally, a deficit was observed in the previous analysis which resulted in higher observed limits. The limits are also stronger compared to the latest CMS result performed with $\SI{36}{\femto\barn}^{-1}$ of data. 

Future analysis will benefit from the increased statistics if they use a similar background estimation method. As the statistics of the dataset increases, the statistical uncertainty associated with the fit will decrease. Therefore, resulting in improvement of analysis sensitivity. However, as the search extends to higher invariant masses, the functional form used may not be suitable and more as-hoc parameters may require to be added. Additionally, other functions may also need to be tested once more. One possible alternative is the use of Gaussian process to model the background~\cite{frate2017modeling}. A Gaussian process provides a generalisation of a particular distribution without being tied to a functional form. Using the method the background distribution can be modelled using Gaussian processes rather than a parametric form. It also allows to incorporate the understanding of the underlying physics to construct the Gaussian process, making it a more physically motivated approach. 


