\chapter{Introduction}\label{chap:intro}

The Standard Model of particle physics has had tremendous success in providing a description of the known universe. There has been excellent agreement between SM perditions and experimental observations from particle physics experiments. The most recent validation of the SM was via the observation of the Higgs Boson by the ATLAS and CMS experiments at the LHC in 2012~\cite{Chatrchyan2012,Aad_2012}.

Despite the successes of the Standard Model, the theory fails to describe a range of observations, mostly from non-collider physics experiments. These signatures hint at the existence of physics beyond the Standard Model. The Standard Model currently includes all of the known particles, however the hint at beyond the standard model indicates that there may be more particles yet to be discovered. Such new particles may not have been discovered as the energy required to produce them might be larger than what is achieved by the current accelerators, and can not be searched for in conventional resonance searches. However, it is possible to detect the effects of such process in low-energy regime in the form of non-resonant signatures. This thesis presents a novel data-driven approach to search for non-resonant phenomena in dilepton final states. 

The thesis is organised the following way. \cref{chap:SM} provides an overview of the Standard Model of particle physics. \cref{chap:bsm} motivates the search for beyond the Standard Model physics, and provides a brief overview of effective field theories. The chapter concludes with an overview of the benchmark contact inter1action model searched for in this thesis. \cref{chap:expSetup} provides an overview of the LHC accelerator and infrastructure, where a detailed outline of the ATLAS detector and it's components are provided. The production and use of Monte Carlo simulations and the reconstruction and identification of physics objects is outlined in \cref{chap:SimReco}. The event selection applied to the data is provided in \cref{chap:eventsel}. \cref{chap:datamc} outlines the data, background and signal samples and their associated uncertainties are described in \cref{chap:sysmc}. The data-driven background estimation method is described in \cref{chap:bkgmodel} and the uncertainties associated with the estimate is described in \cref{chap:uncertBkgmodel}. A detailed description of the statistical analysis is given in \cref{chap:stats}. Finally, the results are presented in \cref{chap:results}. 

\cref{chap:SM,chap:bsm,chap:expSetup,chap:SimReco} provide descriptions of the theoretical and experimental topics that are relevant for this thesis. The chapters were written based on previous literature, which include: papers, technical reports and academic text. The author was one of two analysis contacts for the result presented in this thesis, which included planning and leading team members towards publication of the result. \cref{chap:eventsel,chap:datamc} include work carried out in collaboration with the team, where work by members of the group have been explicitly referenced. Original work by the author in collaboration with the other analysis contact is presented in \cref{chap:bkgmodel,chap:uncertBkgmodel,chap:stats,chap:results}. In addition, the author along with the other analysis contact created and developed the framework used for the background estimation and statistical analysis used in the analysis presented in this thesis. All plots and tables in this thesis were produced by the author, unless otherwise states. 

The author performed the analysis in search for contact interactions in the electron channel at $\sqrt{s} = $ \SI{13}{\tera\electronvolt} and an integrated luminosity of \SI{36.1}{\femto\barn^{-1}}~\cite{EXOT-2016-05}. Additionally, the author was worked in collaboration on a complimentary background estimation procedure for the dilepton resonance search analysis performed  at $\sqrt{s} = $ \SI{13}{\tera\electronvolt} and an integrated luminosity of \SI{139}{\femto\barn^{-1}}~\cite{Aad:2019fac}. These are not presented in the thesis. 