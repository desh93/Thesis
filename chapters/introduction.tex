\chapter{Introduction}\label{chap:intro}

The Standard Model of particle physics has had tremendous success in providing a description of the known universe. There has been excellent agreement between Standard Model predictions and experimental observations from particle physics experiments. The most recent validation of the SM was via the observation of the Higgs boson by the ATLAS and CMS experiments at the LHC in 2012~\cite{Chatrchyan2012,Aad_2012}.

Despite the successes of the Standard Model, the theory fails to describe a range of observations, mostly from non-collider physics experiments. These signatures hint at the existence of physics beyond the Standard Model. The Standard Model currently includes all of the known particles, however physics beyond the standard model indicates that there may be more particles yet to be discovered. Such new particles may not have yet been discovered as the energy required to produce them might be larger than what is achieved by current accelerators, and hence can not be searched for in conventional resonance searches. However, it is possible to detect the effects processes in low-energy regimes in the form of non-resonant signatures. Previously, such searches had relied on the production of Monte Carlo to model the background and signal processes, where samples with very large numbers of events were needed to be produced, resulting in huge demands on the available computing resources. To reduce the strain on resources this thesis presents a novel data-driven approach to search for non-resonant phenomena in dilepton final states. 

The thesis is organised in the following way. \cref{chap:SM} provides an overview of the Standard Model of particle physics. \cref{chap:bsm} motivates the search for beyond the Standard Model physics, and provides a brief overview of effective field theories. The chapter concludes with a description of the benchmark contact interaction model searched for in this thesis. \cref{chap:expSetup} provides an overview of the LHC accelerator and infrastructure, where a detailed outline of the ATLAS detector and its components are provided. The production and use of Monte Carlo simulations and the reconstruction and identification of physics objects are outlined in \cref{chap:SimReco}. The event selection applied to the data is provided in \cref{chap:eventsel}. \cref{chap:datamc} outlines the data, background and signal samples, and their associated uncertainties are described in \cref{chap:sysmc}. The data-driven background estimation method is described in \cref{chap:bkgmodel} and the uncertainties associated with the estimate is described in \cref{chap:uncertBkgmodel}. A detailed description of the statistical analysis is given in \cref{chap:stats}. Finally, the results are presented in \cref{chap:results}. 