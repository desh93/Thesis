\chapter{Object and event selection}\label{chap:eventsel}
The dataset is filtered to select collision events which match the desired final state signature of dielectron or dimuon final states. The cuts are chosen to ensure a good purity of signal electrons or muons in the selected samples, whilst reducing contamination from background processes. A series of object and event selections are made after the reconstruction of the electron and muon events. The selection for the analysis is performed using the SUSYTools package~\cite{SUSYTools} within the Athena framework~\cite{Athena}.

\section{Event selection}\label{sec:selevent}
Events undergo a selection based on the data quality and trigger requirements, requiring events to satisfy a set of baseline "data quality" (DQ) requirements. Events are required to be measured during a period where all detector subsystems were operating at nominal run conditions. Vetos are applied to reject corrupted events based on data quality flags from certain detector subsystems. These requirements are only applied to data. The selected events are then required to have a primary vertex with at least two associated tracks following the definition from~\cref{sec:reconstruction:tracks}. Events in the electron channel are recorded using dielectron triggers, which require the dielectron system to pass the \et threshold between 12 and \SI{24}{\giga\electronvolt} for both electrons, depending on the data taking year. To reduce the trigger rate, additional identification criteria are applied at the trigger level. Single muon triggers are used for the muon channel, which have a \pt thresholds of 26 and \SI{50}{\giga\electronvolt}. The trigger with the \SI{26}{\giga\electronvolt} also requires the muon candidate to be isolated~\cite{Aaboud:2016leb}.

\section{Object Selection}\label{sec:selobejct}
All electrons and muons considered are reconstructed by the algorithms described in  ~\cref{sec:reconstruction}. 

\subsection{Electrons}\label{sec:selee}
The electrons candidates are required to be in the central region of the detector of $\abs{\eta} < 2.47$, to ensure that majority of the electrons from the electromagnetic shower are considered. The transition region $1.37 < \abs{\eta} < 1.52$ between the barrel and the endcap electromagnetic calorimeters are rejected. This requirement vetoes any electrons with poor energy resolution due to the lower cell granularity and significant additional inactive material in the transition region. Additionally, the electron candidates are required to have $\et > \SI{30}{\giga\electronvolt}$ electron candidates are required to have $\et > \SI{30}{\giga\electronvolt}$. All electrons are required to pass the medium likelihood identification criteria as described in \cref{sec:reconstruction:electrons}, which gives a good balance of reduction of other background processes (e.g. QCD, W+Jets) that fake electrons and high signal efficiency. To further reduce the fake background processes, the electrons are required to pass the gradient isolation criteria as described in \cref{sec:reconstruction:electrons}. However, there is a small fraction of fake electrons which pass the selection criteria. Therefore, to model the contribution of fake electrons that arise, a data driven method is used, which is described in \cref{chap:datamc}. The matrix method requires a data set consisting of an enhanced contribution of fake electrons formed using the loose likelihood identification criteria. To ensure the electron candidates originate from the primary vertex the electron candidate track is required to have a $d_0$ significance below 5, and $\abs{z_0 \sin\theta} < \SI{0.5}{\milli\meter}$ for the transverse impact parameter. \cref{sec:reconstruction:tracks} provides a detailed description of the reconstruction of tracks.

\subsection{Muons}\label{sec:selmm}
The muon selection requires muons with $\pt > \SI{30}{\giga\electronvolt}$ and $\abs{\eta} < 2.5$. Muon candidates are required to pass the high-\pt identification criteria, which is designed for analyses in the high \pt regime, ensuring a good momentum resolution at high \pt (\cref{sec:reconstruction:muons}). A requirement to reject "bad muons" is placed on the relative uncertainty of the \emph{q/p} measurement with a \pt dependent cut to ensure better reconstruction properties. A detailed description of the bad muon veto is given below. Similar to electrons a requirement on the transverse impact parameter of $d_0 < 3 \sigma$ and requirement on the longitudinal impact parameter of $\abs{z_0\sin\theta < \SI{4}{\milli\meter}}$ is required. The \emph{FCTightTrackOnly} isolation criteria is used for the muon isolation as described in \cref{sec:reconstruction:muons}.

\subsubsection{Bad muon veto}\label{sec:selmm:badmuon}
The bad muon veto is optimised to reject muons in the tails of the $\sigma_{\pt}/\pt$ distributions, that consists muons with poor reconstruction residing in the tails of the \pt distributions. The efficiency of the veto diminishes with increasing \pt due to the cut on the relative \emph{q/p} measurement. The expected muon momentum resolution is parameterised as a function of \pt defined as~\cite{Aad:2016jkr}:
\begin{equation}
   \frac{\sigma_{\pt}}{\pt} = \frac{r_0}{\pt} \oplus r_1 \oplus r_2 \cdot \pt,
\end{equation}
where the first term models the fluctuations of energy loss in the traversed materials. The second terms accounts for multiple scatterings, local magnetic field fluctuations and local radial displacements Finally, the last term describes the intrinsic resolution effects caused by the spatial resolution of the hit measurements. 

The expected resolution is then parameterised as a function of \pt coinciding with specific regions of the ATLAS detector. The bad muon veto applies a cut on the relative error on the \emph{q/p} measurement of the muon as: 
\begin{equation}
    \frac{\sigma_{q/p}}{q/p} < C(\pt) \cdot \sigma,
 \end{equation}
where $\sigma$ is the expected muon resolution as described above as a function of \pt and $\eta$. C(\pt) is a \pt dependent coefficient equal to 1.8 if $\pt < \SI{1}{\tera\electronvolt}$ which linearly decreases if \pt $> \SI{1}{\tera\electronvolt}$. The C(\pt) cut was optimised to improve the efficiency by 5\% with respect to the previous definition~\cite{EXOT-2016-05}, where a value of $C(\pt) = 2.5$ was chosen~\cite{Aad:2019fac}. 

\subsection{Overlap removal}
The reconstruction algorithms used to define physics objects do not use detector signals exclusively. Therefore, a detector signal used to construct one object can be used to construct another. To remove these ambiguities, the algorithm considers spatially overlapping pairs of objects and elements are removed from the event depending on a defined set of criteria. \cref{tab:sel:overlap} outlines the overlap removal criteria for the analysis selection. 

\begin{table}[]
   \centering
   {
   \begin{tabular}{l|l||c}
 \hline
     Reject & Against & Criteria \\
     \hline
     electron & muon & shared track, $p_{T,1} < p_{T,2}$ \\
     muon & electron & is calo-muon and shared ID track \\
     electron & muon & shared ID track \\
 \hline
   \end{tabular}
   }
   \caption[Overlap removal criteria for the analysis selection]{Overlap removal criteria for the analysis selection}
   \label{tab:sel:overlap}
 \end{table}

\section{Dilepton selection}
All events passing the dilepton selection for the electron and muon channels are required to contain at least two same flavour leptons (electrons or muons). The leading and subleading leptons are then selected depending on their \et (\pt) in the electron (muon) channel. The muon channel imposes an opposite charge requirement. This requirement results in a better \pt reconstruction in muons since muons with misidentified charges will have strongly distorted \pt measurements. However, for the electron channel, no opposite charge requirement is applied, due to the significant loss of efficiency in the high \et regime, and no significant reduction of background processes is expected. Finally, A dilepton invariant mass of $m_{ll} > \SI{130}{\giga\electronvolt}$ is required, where additional restrictions on the invariant mass are placed in the statistical analysis and background estimation. In scenario where events are selected for both the electron and muon channels, the electron channel is preferred over the muon channel due to the better invariant mass resolution in that channel. The rejected overlapping dimuon events only account for a small fraction of the observed events. This cut allows for two independent data sets to be formed for both the channels, allowing for the statistical combination of the two channels. \cref{tab:sel} provides an overview of the full analysis selection. 

\begin{table}[h]
    \centering
    {\begin{tabular}{l|p{5cm}|p{5cm}}
        % \hline
        & \textbf{Electron channel} & \textbf{Muon channel} \\
        \hline
        & \multicolumn{2}{c}{\textbf{Event level}} \\
        \hline
        Trigger & Lowest unprescaled \newline dielectron trigger & Lowest unprescaled \newline single muon trigger \\
        \hline
        Event Cleaning & Applied & Applied \\
        % \emph{Feature} & \emph{Criteria} \\
        \hline
        & \multicolumn{2}{c}{\textbf{Lepton level}} \\
        \hline
        $\abs{\eta}$ range & $\abs{\eta} < 1.37$, $1.52 < \abs{\eta} < 2.47$ & $\abs{\eta} < 2.5$\\
        \hline
        p$_T$ or E$_T$ & $> 30~GeV$ &  $> 30~GeV$ \\
        \hline
        $|d_{0}^{BL}(\sigma)|$ & $< 5$  & $< 3$\\ 
        \hline
        $|\Delta z_{0}^{BL} \sin{\theta}|$ & $< 0.5~mm$ & $< 0.5~mm$ \\
        \hline
        Identification WP & MediumLLH, \newline LooseAndBLayerLLH (Fakes) & High-pT \\
        \hline
        Isolation & Gradient & FCTightTrackOnly \\
        \hline
        & \multicolumn{2}{c}{\textbf{Dilepton level}} \\
        \hline
        Lepton number & $>2$ & $>2$  \\
        \hline
        Object & Highest $E_T$ pair & Highest $p_T$ pair \\
        \hline
        Opposite Sign & Not required & Required \\
        \hline
        $M_{ll}$ & $> 130~GeV$ & $> 130~GeV$ \\

	\end{tabular}}
    \caption[Selection definitions for the analysis selection]{Selection definitions for the analysis selection, including event level, object level and dilepton selection.}
    \label{tab:sel}
  \end{table}

  \clearpage