\pdfbookmark[0]{Frontmatter}{frontmatter}
\pdfbookmark[1]{Title page}{titlepage}
\titlepage[Department of Physics\\ Royal Holloway, University of London]
{A thesis submitted to the University of London\\ for the degree of Doctor of Philosophy}

\cleardoublepage\pdfbookmark[1]{Declaration}{declaration}
\begin{declaration}
    I declare that the work presented in this thesis is my own.
    Where the work of others has been consulted, this has been indicated in the text.

    \cref{chap:SM,chap:bsm,chap:expSetup,chap:SimReco} provide descriptions of the theoretical and experimental topics that are relevant for this thesis. The chapters were written based on previous literature, which include: papers, technical reports and academic text. I was one of two analysis contacts for the result presented in this thesis, which included planning the analysis and leading team members towards publication of the result~\cite{Aad:2020otl}. \cref{chap:eventsel,chap:datamc} include work carried out in collaboration with the team, where work by members of the group have been explicitly referenced. My original work in close collaboration with the other analysis contact is presented in \cref{chap:bkgmodel,chap:uncertBkgmodel,chap:stats,chap:results}. In addition, the other contact and I created and developed the framework used for the background estimation and statistical analysis of the results. All plots and tables in this thesis were produced by me, unless otherwise stated. 

    Furthermore, I performed the statistical analysis in the search for contact interactions in the electron channel at $\sqrt{s} = $ \SI{13}{\tera\electronvolt} with an integrated luminosity of \SI{36.1}{\femto\barn^{-1}}~\cite{EXOT-2016-05}. Additionally, I worked in collaboration on a complementary background estimation procedure for the dilepton resonance search performed  at $\sqrt{s} = $ \SI{13}{\tera\electronvolt} with an integrated luminosity of \SI{139}{\femto\barn^{-1}}~\cite{Aad:2019fac}. These analyses are not discussed explicitly the thesis. 
    \vspace*{1.5cm}
    \begin{flushright}
        Deshan Kavishka Abhayasinghe \\
        July 2020
    \end{flushright}
\end{declaration}


\begin{abstract}\pdfbookmark[1]{Abstract}{abstract}
    A search for new physics with non-resonant signals in dielectron and dimuon final states in mass ranges above \SI{2}{\tera\electronvolt} is presented. The data, corresponding to an integrated luminosity of \SI{139}{\femto\barn^{-1}}, were recorded by the ATLAS experiment in proton--proton collisions at a centre-of-mass energy of \SI{13}{\tera\electronvolt} during Run-2 of the Large Hadron Collider.  A data-driven background extrapolation procedure is utilised to model the contribution from background processes. The benchmark signal model considered is the four-fermion contact-interaction, which results in an enhancement of the dilepton event rate at the TeV mass scale. No significant deviation from the expected background is observed in data. Observed and expected  95\% CL limits on the contact interaction energy scale reach \SI{35.8}{\tera\electronvolt} and \SI{37.6}{\tera\electronvolt}, respectively. In addition, 95\% CL limits on the number of events and visible cross section times branching fraction for new physics processes are provided.   
\end{abstract}
   
\cleardoublepage\pdfbookmark[1]{Acknowledgements}{Acknowledgements}
\begin{acknowledgements}
    First, and foremost, I would like to thank V\'{e}ronique Boisvert, I am eternally grateful for her encouragement, support and patience throughout my PhD. My thanks is extended to Tracey Berry for giving me this amazing opportunity. I would like to thank Glen Cowan for passing down his wisdom on all things statistics. I would also like to thank the ATLAS group at Royal Holloway for many interesting and useful discussions that have helped me with my research.

    I would like to thank Aaron White for working with me as analysis contact over the course of the analysis and putting up with me. I can confidently say this analysis would not be possible without his help. I am thankful for the extensive discussions, never-ending meetings, and arduous debugging sessions we had. 
    
    There were numerous people that have made my time at CERN enjoyable. David and Joe for their friendship. Adam for the many hours we have spent discussing statistical concepts and his continued friendship and support during my PhD. Noam Tal Hod for acting as a surrogate supervisor at CERN and putting up with my endless questions. The dilepton analysis team for the collaboration and the feedback during our many meetings. My flatmates Lewis Wilkins and Eddie Thorpe who kept me sane throughout my time at CERN. Special thanks must go to Lewis, we started our PhD together at Royal Holloway and he has been an invaluable friend throughout my PhD. I do not think I have spent more time with any one person than I have with Lewis during my time at CERN and there has never been a dull moment. 
    
    To Vasilis and Umit, who have been a constant throughout my entire education, thank you for all the crazy adventures and for always being there. Neil and Chanat, my oldest friends, I am grateful for the many dinners, nights out and welcome distractions you have provided me, which have kept me sane. 

    No words can express my gratitude towards my family. Amma, Thatha and Manji, I would not be the person I am today without you. I owe you everything. 

    To Natalia, who has been with me from the start of this journey. You have stayed with me through all of my ups and down, travelling and the countless late nights that I had to spend working. Thank you for all of your love, support and encouragement. $\Sigma\epsilon'~ \alpha\gamma\alpha\pi\acute{\omega}$.
    
\end{acknowledgements}

\cleardoublepage\pdfbookmark[1]{Contents}{toc}\tableofcontents
\cleardoublepage\pdfbookmark[1]{Figures}{tof}\listoffigures
\cleardoublepage\pdfbookmark[1]{Tables}{tot}\listoftables

\dedication{In memory of Sudheera Nanayakkara (Ammi)}

