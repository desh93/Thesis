\usepackage[utf8]{inputenc}
\usepackage[UKenglish]{babel}
\usepackage[numbers,sort&compress]{natbib}

\usepackage{tabularx}
\usepackage{multirow}

\usepackage[dvipsnames]{xcolor} % need to load before pgfplots
\usepackage{pgfplots}
\usepackage{makecell}
\usepgfplotslibrary{groupplots,fillbetween}
\pgfplotsset{compat=1.15}
\pgfplotsset{ticklabel style={font=\footnotesize}}
\pgfplotsset{xlabel style={font=\small,align=right,text width=0.8\figurewidth}}
\pgfplotsset{ylabel style={font=\small,yshift=-1ex,align=right,text width=0.7\figureheight}}
\pgfplotsset{legend style={draw=none,fill=none,font=\footnotesize}}
\pgfplotsset{title style={font=\small}}

\usepgfplotslibrary{external}
\tikzexternalize
\tikzset{%
  external/system call={lualatex \tikzexternalcheckshellescape --halt-on-error --interaction=batchmode --output-directory=./build --jobname "\image" "\texsource"},
  /pgf/images/include external/.code={%
    \includegraphics{build/#1}%
  },  
}

\newlength\figureheight
\newlength\figurewidth
\newlength\subfigwidth

\usepackage{amsmath,amsfonts,amssymb,commath}
\DeclareMathOperator{\cov}{cov}
\DeclareMathOperator{\expectation}{\mathbb{E}}
\DeclareMathOperator*{\argmin}{arg\,min}
\DeclareMathOperator*{\argmax}{arg\,max}
\newcommand{\GP}{\mathrm{GP}\!}
\newcommand{\given}{\,|\,}

\usepackage{subdepth}

%TC:ignore
% Don't double-space display environments
% \BeforeBeginEnvironment{equation}{\vspace{-\baselineskip}\begin{singlespace}}
% \AfterEndEnvironment{equation}{\end{singlespace}\vspace{-.5\baselineskip}\noindent\ignorespaces}
% \BeforeBeginEnvironment{align}{\begin{singlespace}}
% \AfterEndEnvironment{align}{\end{singlespace}\noindent\ignorespaces}
%TC:endignore

\usepackage{pdflscape}
\usepackage{adjustbox}

\usepackage{bm,upgreek}
\renewcommand{\vec}[1]{\boldsymbol{\mathbf{#1}}}
\newcommand{\trans}[1]{{#1}^\mathsf{T}}

\usepackage[pagewise]{lineno}
% Fix for lineno with AMS math environments
\newcommand*\patchAmsMathEnvironmentForLineno[1]{
  \expandafter\let\csname old#1\expandafter\endcsname\csname #1\endcsname
  \expandafter\let\csname oldend#1\expandafter\endcsname\csname end#1\endcsname
  \renewenvironment{#1}%
     {\linenomath\csname old#1\endcsname}%
     {\csname oldend#1\endcsname\endlinenomath}}% 
\newcommand*\patchBothAmsMathEnvironmentsForLineno[1]{%
  \patchAmsMathEnvironmentForLineno{#1}%
  \patchAmsMathEnvironmentForLineno{#1*}}%
\AtBeginDocument{%
\patchBothAmsMathEnvironmentsForLineno{equation}%
\patchBothAmsMathEnvironmentsForLineno{align}%
\patchBothAmsMathEnvironmentsForLineno{multline}%
}

\usepackage[compat=1.1.0]{tikz-feynman}
\tikzset{/tikzfeynman/warn luatex=false,
         /tikzfeynman/momentum/arrow shorten=0.25
}

\usepackage{slashed} % Feynman slash notation

\usepackage[italic]{hepnames}
\def\@shiftlen@anti@gen@bar{1mu} % Hacky fix for Andy's insanity
\usepackage[separate-uncertainty=true,range-phrase={ to },range-units=single,multi-part-units=single,list-units=single,list-final-separator={, and }]{siunitx}
\usepackage{xspace}
\DeclareRobustCommand{\LHC}{LHC\xspace}
\DeclareRobustCommand{\CERN}{CERN\xspace}
\DeclareRobustCommand{\ATLAS}{ATLAS\xspace}
% \DeclareRobustCommand{\APtop}{\overline{t}}
% \DeclareRobustCommand{\APbottom}{\overline{b}}
\DeclareRobustCommand{\ttbar}{\(\Ptop\APtop\)\xspace}
\DeclareRobustCommand{\protonproton}{$\Pproton\Pproton$\xspace}
\DeclareRobustCommand{\CP}{$C \mkern-1.0mu P$\xspace}

\DeclareRobustCommand{\pT}{\ensuremath{p_\text{T}}\xspace}
\DeclareRobustCommand{\mtt}{$m_{\Ptop\APtop}$\xspace}

\def\newarraystrech{0.6}

\newcommand{\errstatsyst}[4]{\ensuremath{{#1}\pm{#2} \, \text{(stat)}\pm{#3}\,\text{(syst)}\,\si{#4}}}
\newcommand{\tol}[3]{\hbox{\rule{0pt}{11pt}\ensuremath{{#1\,}^{+{#2}}_{-{#3}}}}}

\newcommand\tablesize{\fontsize{8pt}{8pt}\selectfont}
% reconstruction section macros
% \newcommand*{\Zee}{\ensuremath{Z \to ee}}
\newcommand*{\et}{\ensuremath{E_{\text{T}}}\xspace}
\newcommand*{\pt}{\ensuremath{p_{\text{T}}}\xspace}
\newcommand{\muhat}{\ensuremath{\left\langle \mu \right\rangle}}

\newcommand*{\etcx}{\ensuremath{E_{\text{T}}^{\text{coneXX}}}\xspace}
\newcommand*{\etcs}{\ensuremath{E_{\text{T}}^{\text{cone20}}}\xspace}
\newcommand*{\etcb}{\ensuremath{E_{\text{T}}^{\text{cone40}}}\xspace}
\newcommand*{\ptcs}{\ensuremath{p_{\text{T}}^{\text{cone20}}}\xspace}
\newcommand*{\ptvcs}{\ensuremath{p_{\text{T}}^{\text{varcone20}}}\xspace}
\newcommand*{\ptcx}{\ensuremath{p_{\text{T}}^{\text{coneXX}}}\xspace}
\newcommand*{\ptvcx}{\ensuremath{p_{\text{T}}^{\text{varconeXX}}}\xspace}
\newcommand*{\ptvcsttva}{\ensuremath{p_{\text{T,TTVA}}^{\text{varcone20}}}\xspace}

% +--------------------------------------------------------------------+
% |  Monte Carlo generators                                            |
% +--------------------------------------------------------------------+
\newcommand*{\ACERMC}{\textsc{AcerMC}\xspace}
\newcommand*{\ACERMCV}[1]{\textsc{AcerMC}~#1\xspace}
\newcommand*{\ALPGEN}{\textsc{Alpgen}\xspace}
\newcommand*{\ALPGENV}[1]{\textsc{Alpgen}~#1\xspace}
\newcommand*{\GEANT}{\textsc{Geant}\xspace}
\newcommand*{\Herwigpp}{Herwig++\xspace}
\newcommand*{\HERWIGpp}{Herwig++\xspace}
\newcommand*{\Herwig}{Herwig\xspace}
\newcommand*{\HerwigV}[1]{Herwig~#1\xspace}
\newcommand*{\HERWIG}{\textsc{Herwig}\xspace}
\newcommand*{\HERWIGV}[1]{\textsc{Herwig}~#1\xspace}
\newcommand*{\JIMMY}{\textsc{Jimmy}\xspace}
\newcommand*{\JIMMYV}[1]{\textsc{Jimmy}~#1\xspace}
\newcommand*{\MADSPIN}{\textsc{MadSpin}\xspace}
\newcommand*{\MADGRAPH}{\textsc{MadGraph}\xspace}
\newcommand*{\MADGRAPHV}[1]{\textsc{MadGraph}~#1\xspace}
\newcommand*{\MGMCatNLO}{\textsc{MadGraph5}\_aMC@NLO\xspace}
\newcommand*{\MGMCatNLOV}[1]{\textsc{MadGraph5}\_aMC@NLO~#1\xspace}
\newcommand*{\MCatNLO}{MC@NLO\xspace}
\newcommand*{\MCatNLOV}[1]{MC@NLO~#1\xspace}
\newcommand*{\AMCatNLO}{aMC@NLO\xspace}
\newcommand*{\AMCatNLOV}[1]{aMC@NLO~#1\xspace}
\newcommand*{\MCFM}{MCFM\xspace}
\newcommand*{\MCFMV}[1]{MCFM~#1\xspace}
\newcommand*{\METOP}{\textsc{MEtop}\xspace}
\newcommand*{\METOPV}[1]{\textsc{MEtop}~#1\xspace}
\newcommand*{\POWHEG}{\textsc{Powheg}\xspace}
\newcommand*{\POWHEGV}[1]{\textsc{Powheg}~#1\xspace}
\newcommand*{\POWHEGBOX}{\textsc{Powheg-Box}\xspace}
\newcommand*{\POWHEGBOXV}[1]{\textsc{Powheg-Box}~#1\xspace}
\newcommand*{\POWPYTHIA}{\POWHEG{}+\PYTHIA}
\newcommand*{\PROTOS}{\textsc{Protos}\xspace}
\newcommand*{\PYTHIA}{\textsc{Pythia}\xspace}
\newcommand*{\PYTHIAV}[1]{\textsc{Pythia}~#1\xspace}
\newcommand*{\SHERPA}{\textsc{Sherpa}\xspace}
\newcommand*{\SHERPAV}[1]{\textsc{Sherpa}~#1\xspace}